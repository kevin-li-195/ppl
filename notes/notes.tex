\documentclass{article}

\title{Probabilistic Programming Languages Notes}

\begin{document}
\maketitle

\section{Notes from reading Practical Probabilistic Programming by Avi Pfeffer}

\subsection{Probabilistic Model as Proof tree}

Examining probabilistic models using a proof-tree construct.
Is this useful? Has this been done? What do we get from this?
What role would proof search play in such a representation?

\section{Type level definition of a probabilistic model}

Could we specify a model using types in the same way that
servant implements their API type? Not quite sure how it's
done, but I think dependencies could be encoded at the type
level, and the specific *way* in which some variable depends
on another could be specified in the type.

What benefit could be wrought from representing a probabilistic
model in a type?

Well, much in the same way as servant, the resulting
model that you would have to implement would construct
a series of monadic (or is applicative more appropriate?)
actions, where the monad is some kind of probability monad.
Except it is different from the simple Random monad in
which only simulation is possible; we could implement
this monad as a free monad (is my understanding of this
correct?), in which case we preserve all
the structure of the computations afterwards (and only use
special combinators to construct such our probabilistic
computations).

\end{document}
