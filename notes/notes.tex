\documentclass{article}

\usepackage{listings}
\usepackage{amsthm,amsmath,amssymb}

\theoremstyle{definition}
\newtheorem{qst}{Question}

\lstset{basicstyle=\footnotesize\ttfamily}

\title{Probabilistic Programming Languages Notes}

\begin{document}
\maketitle

\section{Notes from reading Practical Probabilistic Programming by Avi Pfeffer}

\subsection{Probabilistic Model as Proof tree}

Examining probabilistic models using a proof-tree construct.
Is this useful? Has this been done? What do we get from this?
What role would proof search play in such a representation?

\section{Type level definition of a probabilistic model}

\textbf{Motivation:} Could we specify a model using types in the same way that
servant implements their API type? Not quite sure how it's
done, but I think dependencies could be encoded at the type
level, and the specific *way* in which some variable depends
on another could be specified in the type.

\begin{qst}
    \emph{What benefit could be wrought from representing a probabilistic model in a type?}

    An obvious benefit is that now, you can guarantee that queries of
    the model are type-safe, as you can only query values
    that are literally specified in the type!

    Well, much in the same way as servant's Handler monad, the resulting
    model that you would have to implement would construct
    a series of monadic (or is applicative more appropriate?)
    actions, where the monad is some kind of probability monad.
    Except it is different from the simple Random monad in
    which only simulation is possible; we could implement
    this monad as a free monad (is my understanding of this
    correct?), in which case we preserve all
    the structure of the computations afterwards (and only use
    special combinators to construct such our probabilistic
    computations).

    We could define the Model type inductively. The definition
    could look as follows.

    \begin{lstlisting}
    type Model = Component <|> Component
    -- Missing some initial part corresponding to arguments to cond. prob.
    -- A model is simply is a conditional probability, no?

    type Component = -- probabilistic sub-computations?
    \end{lstlisting}
\end{qst}

\begin{qst}
    \emph{At first glance, Figaro forces the user to choose the
    inference algorithm. Would we instead be able to infer
    whether or not exact inference is possible from the type
    (if we had a clever enough type)?}
\end{qst}

\begin{qst}
    \emph{What combinators should we have?}

    \begin{lstlisting}
probGT :: Prob Double -> Double -> Prob a -> Prob b -> Model (a/b) ?
-- ^ If the probability is greater than some threshold, then a otherwise b.
-- Is it required that a = b?

-- This would actually not be necessary if Prob were a functor, because
-- we could simple fmap (>) ma
    \end{lstlisting}
\end{qst}

\begin{qst}
    \emph{Are Arrows useful for representing conditional probability?
    Is there some kind of datatype representing injective functions?}
\end{qst}

\end{document}
